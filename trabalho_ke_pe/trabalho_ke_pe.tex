\part{Trabalho, Energia Cinética e Energia Potencial}
\section{Conteúdo Importante}
\subsection{Energia Cinética}

Para um objeto com massa $m$ e velocidade $v$, a energia cinética está definida como

\begin{equation}
    K=\frac{1}{2}mv^2
\end{equation}

A energia cinética é um escalar (tem magnitude mas não direção); é sempre um número positivo; e tem unidades do SI de $kg\cdot m^2s^{-2}$. Esta combinação é também conhecida como \textbf{joule}:

\begin{equation}
    1\ joule=1J=1\ kg\cdot m^2s^{-2}
\end{equation}

\subsection{Trabalho}
Quando um objeto se move enquanto um força é exercida sobre ele, então há \textbf{trabalho} a ser feito no objeto pela força.
Se um objeto efetuar um deslocamento $d$ enquanto uma \emph{força constante} \textbf{F} atua nele, a força faz uma quantidade de trabalho igual a

$$
W=F\cdot d=Fdcos\phi
$$

onde $\phi$ é o ângulo entre \textbf{d} e \textbf{F}. O trabalho é também uma grandeza escalar cujas unidades são $N\cdot m$.

O trabalho pode ser negativo; isto acontece quando o ângulo entre a força e o deslocamento é maior que 90^{\circ}. Também pode ser \emph{zero}; isto acontece se $\phi=90^{\circ}$. Para que exista trabalho, a força tem de ter uma componente ao longo (ou oposta) à direção do deslocamento.
Se várias forças (constantes) atuarem na massa enquanto se move ao longo de um deslocamento \textbf{d}, podemos falar do \textbf{trabalho resultante} feito pelas forças,

$$
\begin{aligned}
    W_{R}&=F_1\cdot d+F_2\cdot d+F_3\cdot d...+F_n\cdot d\\
    &=(\sum F)\cdot d \\
    &=F_{R}\cdot d
\end{aligned}
$$

Se a força que atua sobre o objeto não é constante enquanto o objeto se move, então devemos calcular um integral para encontrar o trabalho realizado.
Supondo que o objeto se mova ao longo de uma linha reta (digamos, ao longo do eixo $x$, de $x_i$ a $x_f$) enquanto
uma força cujo componente $x$ é $F_x(x)$ atua sobre ele. (Ou seja, conhecemos a força $F_x$ como uma função
de $x$.) Então, o trabalho realizado é

$$
\begin{equation}
    W=\int_{x_i}^{x_f}F_x(x)dx
\end{equation}
$$

Finalmente, podemos dar a expressão general para o trabalho feito por uma força. Se um objeto se move de $r_i=x_ii+y_ij+z_ik$ a $r_f=x_fi+y_fj+z_fk$ enquanto a força $F(r)$ atua sobre ele, o trabalho feito é:

$$
    \begin{equation}
        W=\int_{x_i}^{x_f}F_x(r)dx+\int_{y_i}^{y_f}F_y(r)dy+\int_{z_i}^{z_f}F_z(r)dz
    \end{equation}
$$

onde os integrais são calculados ao longo do caminho traçado pelo objeto. Esta expressão pode ser abreviada por

$$
W=\int_{r_i}^{r_f}F\cdot dr
$$

A gravidade funciona em objetos que se movem verticalmente. Pode-se mostrar que se a altura de um objeto mudou numa quantidade $\Delta y$, então a gravidade fez uma quantidade de trabalho igual a 
$$
W_{g}=-mg\Delta y
$$

sem qualquer influência do deslocamento horizontal. Observe-se o sinal de menos aqui; se o objeto aumenta em altura, moveu-se de forma oposta à força da gravidade.

\subsection{Molas}