\section{Equilíbrio Estático}
\subsection{Conteúdo Importante}

Neste capítulo, estudamos um caso especial da dinâmica de objetos rígidos abrangidos nos últimos dois capítulos. É o caso especial (muito importante!) pnde o centro de massa do objeto não tem movimento e o objeto não está em rotação.

\subsubsection{Condições para Equilíbrio de um Corpo Rígido}
Para um objeto rígido que não se move, temos as seguintes condições:

\begin{itemize}
    \item a soma (vetorial) das forças externas no corpo rígido deve ser zero:
    \begin{equation}
        \sum F =0
    \end{equation}
    Quando esta condição é satisfeita, diz-se que o objeto está em \textbf{equilíbrio translacional}.
    \item a soma das torques externas no corpo rígido deve ser zero:
    \begin{equation}
        \sum \tau = 0
    \end{equation}
    Quando esta condição é satisfeita, dizemos que o objeto está em \textbf{equilíbrio rotacional}.
\end{itemize}

Quando ambas as condições acimas são satisfeitas, diz-se que o corpo está em \textbf{equilíbrio estático}.

\subsubsection{Exemplos de Corpos Rígidos em Equilíbrio Estático}
Estratégia para resolver problemas em equilíbrio estático:
\begin{enumerate}
    \item determinar todas as forças que atuam no corpo rígido. Essas forças virão de outros objetos com o qual o corpo está em contacto (suportes, paredes, chão, massas em repouso sob ele) bem como a gravidade;
    \item desenhar um diagrama e colocar toda a informação existente sobre essas forças: os pontos do corpo no qual elas agem, as magnitudes, as direções;
    \item escrever as equações para o equilíbrio estático. Para a equação da torque, vhá opção de escolha de onde colocar o eixo; ao fazer essa escolha, há que pensar em qual ponto tornaria as equações resultantes mais simples;
    \item resolver as equações.
\end{enumerate}