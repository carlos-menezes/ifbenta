\documentclass[a4paper, 12pt]{article}
\setcounter{secnumdepth}{0}

\usepackage{epigraph}

\usepackage[portuges]{babel}
\usepackage[utf8]{inputenc}
\usepackage{amsmath}
\usepackage{indentfirst}
\usepackage{graphicx}
\usepackage{multicol,lipsum}
\usepackage{caption}
\usepackage{listings}
\usepackage{subcaption}
\usepackage{biblatex}

\setcounter{tocdepth}{5}
\setcounter{secnumdepth}{5}

\usepackage{hyperref}
\hypersetup{
    colorlinks,
    citecolor=black,
    filecolor=black,
    linkcolor=black,
    urlcolor=black
}

\newtheorem{definition}{Definição}[section]
\newtheorem{remark}{Observação}

\title{Mecânica e Ondas\\ \Large{Sebenta de Apoio ao Estudo}}
\author{Carlos Menezes}
\date{4 de maio de 2021}

\begin{document}

\maketitle

\newpage

\tableofcontents

\newpage
\newpage

\part{Cinemática e Dinâmica}

\newpage

\section{Movimento em Uma Dimensão}
\subsection{Conteúdo Importante}
\subsubsection{Posição, Tempo e Deslocamento}

Começamos o estudo do movimento considerando objetos que são muito pequenos em comparação com o tamanho de seu movimento no espaço. Quando podemos lidar com um objeto desta forma, referimo-nos a ele como uma \textbf{partícula}. Neste capítulo, lidamos com o caso em que uma partícula se move ao longo de uma linha reta.

A localização de uma partícula é especificada pela sua coordenadas, denotada por $x$ ou $y$. À medida que a partícula se move, a sua coordenada muda com o tempo, $t$. A mudança de posição da partícula de $x_1$ para $x_2$ é o deslocamento $\Delta x$, com $\Delta x = x_2 - x_1$.

\subsubsection{Velocidade Média e Celeridade Média}
Quando uma partícula sofre um deslocamento $\Delta x$ num intervalo de tempo $\Delta t$, a sua \textbf{velocidade média} para esse intervalo de tempo é

\begin{equation}
    \bar{v}=\frac{\Delta x}{\Delta t}=\frac{x_2-x_1}{t_2-t_1}
\end{equation}

A \textbf{celeridade média} de uma partícula é o valor absoluto da velocidade média e é dado por

\begin{equation}
    \bar{c}=\frac{\text{Distância percorrida}}{\Delta t}
\end{equation}

\subsubsection{Velocidade Instantânea e Celeridade Instantânea}
Podemos responder à pergunta \emph{"o quão rápido se move a partícula no instante t?"} descobrindo a sua velocidade instantânea. Este é o caso em que o tempo considerado na velocidade média, $\Delta t$, converge para \emph{zero}:

\begin{equation}
    v=\lim_{\Delta t \to 0}\frac{\Delta x}{\Delta t}
\end{equation}

A \textbf{celeridade instantânea} é o valor absolute (magnitude) da velocidade instantânea.

Num gráfico posição-tempo ($x\ vs.\ t$) para uma partícula que se move, a velocidade instantânea é o declive da tangente à curva em qualquer ponto.

\subsubsection{Aceleração}
Quando a velocidade de uma partícula se altera, diz-se que a partícula sofre aceleração.

Se a velocidade de uma partícula se altera de $v_1$ para $v_2$ durante o intervalo de tempo $t_1$ a $t_2$, a \textbf{aceleração média} é definida como

\begin{equation}
    \bar{a}=\frac{\Delta v}{\Delta t}=\frac{x_2-x_1}{t_2-t_1}
\end{equation}

Tal como a velocidade, é importante racicionar sobre a \textbf{aceleração instantânea}, dada por

\begin{equation}
    a=\lim_{\Delta t \to 0}\frac{\Delta v}{\Delta t}=\frac{dv}{dt}
\end{equation}

Se a aceleração $a$ for positiva, a velocidade instantânea $v$ está a aumentar; se $a$ for negativo, então $v$ está a diminiur.

\subsubsection{Aceleração Constante}
Um caso útil \emph{especial} do movimento acelerado é aquele em que a aceleração $a$ é constante. Para este caso, é possível mostrar que as seguintes equações são verdadeiras:

\begin{equation}
    v=v_0+at
\end{equation}
\begin{equation}
    x=x_0+v_0t+\frac{1}{2}at^2
\end{equation}
\begin{equation}
    v^2=v_0^2+2a(x-x_0)
\end{equation}
\begin{equation}
    x=x_0+\frac{1}{2}(v_0+v)t
\end{equation}

Nas equações acimas, a partícula tem posição $x_0$ e velocidade $v_0$ no tempo $t=0$; tem posição $x$ e velocidade $v$ no tempo $t$.
Atente-se que as equações acima apenas são válidas para cases em que a \text{aceleração é constante}!

\subsubsection{Queda Livre}
Um objeto lançado para cima ou para baixo perto da superfície terrestre tem aceleração constante, direcionada para baixo, de magnitude $9.8\ ms^{-2}$. Este número é denotado por $g$.
É preciso ter um cuidado redobrado com o sinal; se no sistema de coordenadas adotado o eixo $y$ aponta para cima, a aceleração de um objeto em queda livre é

\begin{equation}
    a_y=-9.80\ ms^{-2}=-g
\end{equation}

Aqui, assume-se que o ar não tem efeito no movimento do objeto em queda livre. É importante saber que a aceleração do objeto é \textbf{sempre} $9.80\ ms^{-2}$, esteja ele a subir, descer ou na altura máxima... sempre!

\newpage

\section{Movimento em Duas e Três Dimensões}
\subsection{Conteúdo Importante}
\subsubsection{Posição}

Em três dimensões, a posição de uma partícula é especificada pelo seu \textbf{vetor posição}, \textbf{r}:

\begin{equation}
    r=x\mathbf{i}+y\mathbf{j}+z\mathbf{k}
\end{equation}

Se, durante o intervalo $\Delta t$ o vetor posição da partícula muda de $r_1$ para $r_2$, o deslocamento $\Delta r$ para esse intervalo de tempo é

\begin{equation}
    \Delta r=r_1-r_2
\end{equation}
\begin{equation}
    =(x_2-x_1)i+(y_2-y_1)j+(z_2-z_1)k
\end{equation}

\subsubsection{Velocidade}

Se uma partícula se move ao longo de um deslocamento $\Delta r$ num intervalo de tempo $\Delta t$, então a sua velocidade média para esse intervalo é

\begin{equation}
    \bar{v}=\frac{\Delta r}{\Delta t}=\frac{\delta x}{delta t}i+\frac{\delta y}{delta t}j+\frac{\delta z}{delta t}k
\end{equation}

Uma quantidade mais interessante é a velocidade \emph{instantânea} \textbf{v}, que é o limite da velocidade média quando se diminiu o intervalo de tempo $\Delta t$ para \emph{zero}. É, portanto, a derivada do vetor posição $r$ em ordem ao tempo:

\begin{equation}
    v=\frac{dr}{dt}
\end{equation}
\begin{equation}
    =\frac{d}{dt}(xi+yj+zk)
\end{equation}
\begin{equation}
    =\frac{dx}{dt}i+\frac{dy}{dt}j+\frac{dz}{dt}k
\end{equation}

A velocidade instantânea $v$ de uma partícula é sempre tangente à trajetória da partícula.

\subsubsection{Aceleração}
Se a velocidade de uma partícula muda por $\Delta v$ num periodo de tempo $\Delta t$, a aceleração média $\bar{a}$ para esse período é

\begin{equation}
    \bar{a}=\frac{\Delta v}{\Delta t}=\frac{\Delta v_x}{\Delta t}i+\frac{\Delta v_y}{\Delta t}j+\frac{\Delta v_z}{\Delta t}k
\end{equation}

Uma quantidade mais interessante é a aceleração \emph{instantânea} \textbf{a}, que é o limite da aceleração média quando se diminiu o intervalo de tempo $\Delta t$ para \emph{zero}. É, portanto, a derivada do vetor velocidade $v$ em ordem ao tempo:

\begin{equation}
    v=\frac{dv}{dt}
\end{equation}
\begin{equation}
    =\frac{d}{dt}(v_xi+v_yj+v_zk)
\end{equation}
\begin{equation}
    =\frac{dv_x}{dt}i+\frac{dv_y}{dt}j+\frac{dv_z}{dt}k
\end{equation}

\subsubsection{Aceleração Constante em Duas Dimensões}
Quando a aceleração $a$ (para um movimento em duas dimensões) é constante, existem duas equações para descrever as coordenadas $x$ e $y$, semelhantes a equações vistas no capítulo anterior. Nas equações seguintes, o movimento da partícula começa em $t=0$; a posição inicial da partícula é dada por

$$
r_0=x_0i+y_0j
$$

e a sua velocidade inicial é dada por

$$
v_0=v_{0x}i+v_{0y}j
$$

e o vetor $a=a_xi+a_yj$ é \emph{constante}.


\begin{equation}
    v_x=v_{0x}+a_xt \qquad v_y=v_{0y}+a_yt
\end{equation}
\begin{equation}
    x=x_{0}+v_{0x}t+\frac{1}{2}a_xt^2 \qquad y=y_{0}+v_{0y}t+\frac{1}{2}a_yt^2
\end{equation}
\begin{equation}
    v_x^2=v_{0x}^2+2a_x(x-x_0) \qquad v_y^2=v_{0y}^2+2a_y(y-y_0)
\end{equation}
\begin{equation}
    x=x_{0}+\frac{1}{2}(v_{0x}+v_x)t \qquad y=y_{0}+\frac{1}{2}(v_{0y}+v_y)t
\end{equation}


\subsubsection{Movimento do Projétil}

Quando uma partícula se move num plano vertical durante uma queda livre, a sua aceleração é constante; tem magnitude 9.8 $ms^{-2}$ e está direcionada para baixo. Se as suas coordenadas são dadas por um eixo horizontal $x$ e um eixo vertical $y$ direcionado para cima, então a aceleração do \textbf{projétil} é

\begin{equation}
    a_x=0 \qquad a_y=-9.8\ ms^{-2}=-g
\end{equation}

Para um projetil, a aceleração horizontal é zero!

O movimento do projétil é um caso especial de aceleração constante, pelo que se usam as equações vistas no capítulo anterior.

\subsubsection{Movimento Circular Uniforme}
Quando uma partícula se move ao longo de uma trajetória circular a uma \emph{velocidade constante}, diemos que a partícula tem um \textbf{movimento circular uniforme}. Apesar de velocidade não mudar, \emph{a partícula está a acelerar} porque a sua velocidade $\mathbf{v}$ está a mudar de \emph{direção}.

A aceleração da partícula está direcionada para o centro do círculo e tem magnitude

\begin{equation}
    a=\frac{v^2}{r}
\end{equation}

onde \emph{r} é o raio da trajetória circular e $v$ é a velocidade (constante) da partícula. Dizemos que a partícula em movimento circular uniforme tem \textbf{aceleração centrípeta} por causa da direção da aceleração (para o centro da trajetória).
Se a partícula fizer, repetidamente, uma trajetória circular completa, então é útil falar no tempo $T$ que ela leva a completar uma volta. A isto chama-se o \textbf{período} do movimento. O período é dado por:

\begin{equation}
    T=\frac{2\pi r}{v}
\end{equation}

\subsubsection{Movimento Relativo}
A velocidade de uma partícula depende de quem está a fazer essa medida; como veremos mais tarde, é perfeitamente válido considerar observadores "móveis" que carregam seus próprios relógios e coordenam sistemas com eles, ou seja, eles fazem medições de acordo com seu próprio \textbf{referencial}; isto é, um conjunto de coordenadas cartesianas que podem estar em movimento em relação a outro conjunto de coordenadas. Aqui, vamos supor que os eixos nos diferentes sistemas permanecem paralelos um ao outro; ou seja, um sistema pode se mover (transladar), mas não girar em relação a outro.

Supondo que os observadores nos \emph{frames} A e B medem a posição de um ponto $P$. Então, se tivermos as definições

$$
\begin{aligned}
    r_{PA}&=\text{posição de P medida por A} \\
    r_{PB}&=\text{posição de P medida por B} \\
    r_{BA}&=\text{posição da origem de B medida por A} \\
\end{aligned}
$$

temos as relações:

\begin{equation}
    r_{PA}=r_{PB}+r_{BA}
\end{equation}

\begin{equation}\label{eq:velocidade-relativa}
    v_{PA}=v_{PB}+v_{BA}
\end{equation}

As Leis de Newton (próximo capítulo) aplicam-se a tais \textbf{referências de inércia}. Observadores em cada um desses \emph{frames} concordam no valor da aceleração da partícula.

Entre outros lugares, a Equação \ref{eq:velocidade-relativa} é usada em problemas onde um objeto como um avião ou barco tem uma velocidade conhecida no \emph{frame} de (em relação a) um meio como o ar ou a água que por si só se move em relação ao solo estacionário; podemos então encontrar a velocidade do avião ou barco em relação ao solo a partir da soma do vetor na Equação \ref{eq:velocidade-relativa}.

\newpage

\section{Forças}
\subsection{Conteúdo Importante}
\subsubsection{Primeira Lei de Newton}

Com as Leis de Newton, começamos o estudo de como o movimento ocorre no mundo real. O estudo das causas do movimento é chamado de dinâmica ou mecânica. A relação entre força e a aceleração foi dada por Isaac Newton em suas três leis do movimento, que formam o base da física elementar. Embora a formulação da física de Newton tivesse que ser substituída mais tarde, para lidar com o movimento em velocidades comparáveis à velocidade da luz e para o movimento no escala de átomos, é aplicável a situações cotidianas e ainda é a melhor introdução para as leis fundamentais da natureza. O estudo das leis de Newton e suas implicações é muitas vezes chamada de mecânica newtoniana ou clássica.

As partículas aceleram porque sofrem ação de forças. Na ausência de forças, uma partícula não acelera, movendo-se com uma velocidade constante.

\begin{definition}[Primeira Lei de Newton.] Considerando um corpo no qual não hã ação de forças. Então, se esse mesmo corpo estiver em repouso, permanecerá em repouso, e se estiver se movendo com velocidade constante, continua a mover-se nessa velocidade.
\end{definition}

\subsubsection{Segunda Lei de Newton}
Experiencias demonstram que os objetos têm uma propriedade chamada \textbf{massa} que mede como o seu movimento é influenciado por forças. A Segunda Lei de Newton é uma relação entre a \textbf{força resultante} ($F$) agindo sobre uma massa $m$ e sua aceleração $a$.

\begin{definition}[Segunda Lei de Newton] $\sum F=ma$
\end{definition}

A relação acima é uma relação \emph{vetorial}, pelo que em duas dimensões, esta equação implica:

\begin{equation}\label{eqn:2newton}
    \sum F_x=ma_x \qquad \sum F_y=ma_y
\end{equation}

A unidade de força no SI é $kg\ ms^{-2}$, abreviada em newton ($N$). Deste modo,

\begin{equation*}
    1\ newton = 1 N = 1\ kg\ ms^{-2}
\end{equation*}

\subsubsection{Exemplos de Forças}
A Terra exerce uma força gravitacional com sentido para baixo em todas as massas perto da sua superfície. Esta força é conhecida como o \textbf{peso}, $F_g$, e a sua magnitude é dada por

\begin{equation}
    F_g=mg    
\end{equation}

Uma corda sob tensão exerce uma força nos objetos que estão apensos em cada uma das pontas. As forças estão direcionadas para dentro ao longo do comprimento da corda. A tensão é simbolizada pela letra $T$.

Uma superfície sólida exerce uma força em qualquer massa com a qual esteja em contacto. De modo geral, a força da superfície terá uma componente perpendicular/normal, denominada \textbf{força normal} da superfície. A superfície pode também exercer uma força paralela: a força de fricção.

\subsubsection{Terceira Lei de Newton}
Esta lei é popularmente enunciada como "lei da ação-reação", mas na verdade lida com as forças entre dois objetos.

\begin{definition}[Terceira Lei de Newton.] Considerando dois objetos A e B. A força que o objeto A exerce no objeto B é igual e oposta à força que o objeto B exerce no objeto A: $F_{AB}=-F_{BA}$
\end{definition}

\subsubsection{Aplicação das Lei de Newton}
Uma dica útil apra problemas que envolvem mais do que uma força é desenhar um diagrama que evidencia as massas individuais no problema, juntamento com os vetores que mostram as direções e magnitudes das forças individuais. A estes diagramas é dado o nome de \textbf{diagramas de corpo-livre}.

\subsubsection{Atrito}
Forças que são conhecidas coletivamente como “forças de atrito” estão ao nosso redor na vida diária. Na física elementar, discutimos a força de atrito conforme ela ocorre entre dois objetos cujas
superfícies estão em contato e deslizam uma contra a outra. Se, em tal situação, um corpo não se mover enquanto uma força $F$ age sobre ele, então
as forças de \textbf{atrito estático} opõem-se à força $F$, pelo que a força resultante será zero. Empiricamente,
descobre-se que esta força pode ter um valor máximo dado por:

$$
f_s^{max}=\mu_sN
$$

onde $\mu_s$ é o \textbf{coeficiente de atrito estático} para as duas superfícies e $N$ é a força normal entre as duas.
Se um objeto está em movimento em relação a outro, então existe uma força de \textbf{atrito cinético} entre os dois objetos. A direção desta força é tal que se opõe ao movimento e a sua magnitude é dada por

$$
f_k=\mu_kN
$$

onde $N$ é a força normal entre os objetos e $\mu_k$ o \textbf{coeficiente de atrito cinético} para as duas superfícies.

\subsubsection{Revisitando o Movimento Circular Uniforme}

Como visto em capítulos anteriores, quand oum objeto está em movimento circular uniforme, movendo-se em forma de círculo de raio \emph{r} com velocidade \emph{v}, a aceleração (centrípeta) é dirigida para o centro do círculo e tem magnitude

\begin{equation}
    a_{c}=\frac{v^2}{r}
\end{equation}

Desta maneira, da Segunda Lei de Newton, a força resultante que atua neste objeto tem também de ser direcionada para o centro da trajetória e ter magnitude

\begin{equation}
    F_{c}=\frac{mv^2}{r}
\end{equation}

Tal força é chamada \textbf{força centrípeta}.

\subsubsection{Lei da Gravitação Universal}
A força gravítica é uma das forças fundamentais da natureza. Todas as massas exercem uma força gravitacional atrativa umas sobre as outras, mas para a maioria dos objetos a força é tão pequena que podemos ignorá-la.

A Lei da Gravitação Universal diz que para duas massas $m_1$ e $m_2$, separadas por uma distância $r$, a magnitude da força gravitacional é

\begin{equation}\label{eq:lei_da_gravidade}
    F=G\frac{m_1m_2}{r^2} \qquad onde \qquad G=6.67\times 10^{-11}\frac{N\cdot m^2}{kg^2}
\end{equation}

% \section{Exercícios}
% \subsection{Segunda Lei de Newton}
\textbf{1.} Uma massa de 3.0kg sofre uma aceleração dada por $a=(2.0i + 5.0j)ms^{-2}$. Qual é a força resultante e a sua magnitude?
\linebreak
A Segunda Lei de Newton diz-se que a força resultante numa massa $m$ é $\sum F = ma$. Portanto,

$$
\begin{aligned}
    F_{R}&=ma \\
        &=(3.0\ kg)(2.0i+5.0j)ms^{-2} \\
        &=(6.0i + 15j)N
\end{aligned}
$$

A magnitude da $F_{R}$ é dada pelo cálculo da norma do vetor $F_{R}$:

$$
\begin{aligned}
    F_{R}=\sqrt{(6.0N)^2+(15.0N)^2}=16N
\end{aligned}
$$

\textbf{2.} Enquanto duas forças agem sobre ela, uma massa de $m=3.2kg$ move-se com uma velocidade constante $(3ms)i-(4ms)j$. Uma dessas forças é $F_1=(2N)i+(-6N)j$. Qual é a outra força?
\linebreak
A Segunda Lei de Newton diz-se que a força resultante numa massa $m$ é $\sum F = ma$. Portanto, $\sum F = ma \equiv F_1+F_2=ma$. Neste caso, a velocidade é constante, pelo que $a=0$. Portanto,

$$
\begin{aligned}
    F_1+F_2&=0 \\
    \implies F_2&=-F_1 \\
    \implies F_2&=-[(2N)i+(-6N)j] \\
    \implies F_2&=(-2N)i+(6N)j
\end{aligned}
$$

\textbf{3.} Um objeto de massa $m=4.0\ kg$ tem uma velocidade de $3.0i\ ms^{-1}$ num dado instante. Oito segundos depois, a sua velocidade é $(8.0i+10.0j)ms^{-1}$. Assumindo que o objeto foi sujeito a uma força resultante constante, encontre (a) as componentes da força e (b) a sua magnitude. \\
\linebreak
\textbf{(a)} É-nos dito que a força resultante que age na massa foi constante. Logo, sabemos que a sua aceleração foi igualmente constante, we podemos usar resultados de capítulos anteriores. São facultadas as velocidades inicias e final, pelo que é possível calcular as componentes da aceleração:

$$
a_x=\frac{\Delta v_x}{\Delta t}=\frac{[(8.0\ ms^{-1})-(3.0\ ms^{-1})]}{(8.0\ s)}=0.63\ ms^{-2}
$$

$$
a_y=\frac{\Delta v_y}{\Delta t}=\frac{[(1.0\ ms^{-1})-(0.0\ ms^{-1})]}{(8.0\ s)}=1.3\ ms^{-2}
$$

Visto que a massa é facultada no enunciado, é possível calcular as componentes da força resultante:

$$
\begin{aligned}
    F_x=ma_x=(4.0\ kg)(0.63\ ms^{-2})=2.5N \\
    F_y=ma_y=(4.0\ kg)(1.3\ ms^{-2})=5.0N \\
\end{aligned}
$$

\textbf{(b)} A magnitude da força resultante é obtida através da seguinte computação:

$$
F=\sqrt{F_x^2+F_y^2}=\sqrt{(2.5\ N)^2+(5.0\ N)^2}=5.6\ N
$$

A direção $\theta$ da força $F$ é dada por

$$
\tan \theta = \frac{F_y}{F_x} = 2.0 \quad \implies \quad \theta = \arctan(2.0)=63.4^{\circ}
$$

\textbf{4.} Cinco forças atuam sobre a caixa de massa $4.0\ kg$ na Figura \ref{fig:4caixa}. Encontre a aceleração da caixa (a) na notação unidade-vetor e (b) a sua magnitude e direção.
\linebreak
\begin{figure}[h!]
    \centering
    \includegraphics[width=0.5\textwidth]{forças/fig/ex4.png}
    \caption{Cinco forças atuam num caixa de massa $4.0\ kg$.}
    \label{fig:4caixa}
\end{figure}


\textbf{(a)} Existe a necessidade de separar as forças nas suas duas componentes $x$ e $y$. A Segunda Lei de Newton será utilizada para computar o valor da aceleração da caixa.

$$
\begin{aligned}
    \sum F_x&=-11N+3N+14\cos(30º) \\
    &=4.1N
\end{aligned}
$$

$$
\begin{aligned}
    \sum F_y&=+5N-17N+14\sin(30) \\
    &=-5.0N
\end{aligned}
$$

Logo, tem-se que a força resultante é

$$
\sum F=(4.1N)i+(-5.0N)j
$$

Utilizando a Equação \ref{eqn:2newton}, temos:

$$
    \sum a_x=\frac{\sum F_x}{m}=\frac{(4.1\ N)}{4.0\ kg}=1.0\ ms^{-2} \\
    \sum a_y=\frac{\sum F_y}{m}=\frac{(-5.0\ N)}{4.0\ kg}=-1.2\ ms^{-2} \\
$$


O valor da aceleração em notação vetorial é, então,

$$
a=(1.0i-1.2j)ms^{-2}
$$

\textbf{(b)} A aceleração encontrada em (a) tem magnitude

$$
a=\sqrt{a_x^2+x_y^2}=\sqrt{(4.0\ ms^{-2})^2+(-1.2\ ms^{-2})^2}=1.6\ ms^{-2}
$$

A direção $\theta$ do vetor aceleração é dada por

$$
\tan \theta = \frac{a_y}{a_x} = -1.2 \quad \implies \quad \theta = \arctan(-1.2)=-50^{\circ}
$$

Neste caso, tendo em conta que $a_y$ é negativo e $a_x$ é positivo, a escolha de $\theta = -50^{\circ}$ está correta.

\subsection{Exemplos de Forças}
\textbf{5.} Encontre a tensão em cada uma das cordas da Figura \ref{fig:5caixa}.
\linebreak
\begin{figure}[h!]
    \centering
    \includegraphics[width=0.5\textwidth]{forças/fig/ex5.png}
    \caption{Massas suspensas por cordas.}
    \label{fig:5caixa}
\end{figure}

\textbf{(a)} Seja $m_1$ a massa correspondente à bola representada na Figura \ref{fig:5caixa} (a). A força gravítica atua para baixo com uma força de magnitude $m_1g$. A corda vertical puxa "para cima" com uma força de magnitude $T_3$. Tendo em conta que a massa pendurada não tem aceleração, verifica-se que $T_3=m_1g$. Portanto, o valor de $T_3$ é dado por:

$$
T_3=m_1g=(5.0\ kg)(9.9\ ms^{-2})=49\ N
$$

Considerando agora o ponto de união das três cordas. Esse ponto não tem qualquer aceleração, pelo que o resultado da força resultante deverá ser zero. As componentes verticais e horizontains dessas forças somam a zero, separadamente.

$$
    \begin{cases}
        -T_1\cos(40^{\circ})+T_2\cos(50^{\circ})=0 \\
        T_1\sin(40^{\circ})+T_2\sin(50^{\circ})-T_3=0
    \end{cases}
$$

A primeira equação, a soma das componentes horizontais, dá-nos $T_2=1.19T_1$.

A segunda equação,  a soma das componentes verticais, dá-nos o valor  $T_1=31.5\ N$.

Sabendo o valor de $T_1$, $T_2=1.19T_1\ \implies \ T_2=37.5\ N$

As tensões no sistema (a) são, portanto:

$$
T_1=31.5\ N \qquad T_2=37.5\ N \qquad T_3=49\ N
$$

\textbf{(b)} A força resultante na massa pendurada, $m_2$, tem de ser 0, porque não há aceleração. Visto que a gravida "puxa para baixo" com uma força $m_2g$ e a corda vertical "puxa para cima" com uma força $T_3$, sabemos que

$$
T_3-F_g=0 \implies T_3=F_g \implies T_3=m_2g \\
\begin{aligned}
    T_3&=m_2g
    &=(10\ kg)\cdot(9.8\ ms^{-2})=98\ N
\end{aligned}
$$

Considerando agora as forças que atuam no ponto onde todas as forças se encontram. Novamente, como não há aceleração nesse ponto, tanto as componentes verticais como horizontais somam a zero.

No que toca às forças horizontais, temos:

$$
-T_1\cos(60^{\circ})+T_2=0 \qquad \implies \qquad T_2=T_1\cos(60^{\circ}) \\
$$

A soma das forças verticais é dada por:

$$
T_1\sin(60^{\circ})-T_3=0 \qquad \implies \qquad T_1=\frac{T_3}{\sin(60^{\circ})}=113\ N
$$

Podemos, depois, obter $T_2$:

$$
T_2=T_1\cos(60^{\circ})=(133\ N)\cos(60^{\circ})=56.6\ N
$$

As tensões no sistema (b) são, portanto:

$$
T_1=133\ N \qquad T_2=56.6\ N \qquad T_3=98\ N
$$

\textbf{6.} Um bloco de massa $m=2.0\ kg$ está suspenso em equilíbrio num encosta que faz um ângulo $\theta=60^{\circ}$ pela força horizontal $F$, como mostrado na Figura \ref{fig:6plano}. (a) Determine o valor de $F$, a magnitude de $F$. (b) Determine a força normal exercida pelo plano inclinado no bloco (ignore o atrito).
\linebreak
\begin{figure}[h!]
    \centering
    \includegraphics[width=0.5\textwidth]{forças/fig/ex6.png}
    \caption{(a) Bloco em repouso numa rampa sem atrito por uma força horizontal. (b) Forças a atuarem no bloco.)}
    \label{fig:6plano}
\end{figure}

\textbf{(a)} Fazer um diagrama de das forças que atuam no corpo torna-se essencial para este problema, daí a inclusão da \ref{fig:6plano} (b). Muitas vezes, para problemas envolvendo um bloco num plano inclinado, é mais fácil usar as componentes da $F_g$ ao longo desse mesmo plano inclinado e perpendicular a ele. Para este problema, isso não torna as coisas mais fáceis, uma vez que não há movimento no plano inclinado.

Do enunciado, retira-se a informação de que o block está em equilíbrio, pelo que não tem aceleração e as forças que em si atuam somam a zero. Este facto permite-nos escrever:

$$
N\sin(30^{\circ})-F_g=0 \qquad \implies \qquad N=\frac{F_g}{\sin(30^{\circ})}
$$

$$
N=\frac{(2.0\ kg)\cdot(9.8\ ms^{-2})}{\sin(30^{\circ})}=39.2\ N
$$

O resultado da força resultante horizontal também é nulo, logo podemos escrever:

$$
F-N\cos(30^{\circ})=0 \qquad \implies \qquad F=N\cos(30^{\circ})=33.9\ N
$$

\textbf{(b)} O resultado da força Normal foi encontrado anteriormente, $N=39.2\ N$.

\newpage

\section{Trabalho, Energia Cinética e Energia Potencial}
\subsection{Conteúdo Importante}
\subsubsection{Energia Cinética}

Para um objeto com massa $m$ e velocidade $v$, a energia cinética está definida como

\begin{equation}
    K=\frac{1}{2}mv^2
\end{equation}

A energia cinética é um escalar (tem magnitude mas não direção); é sempre um número positivo; e tem unidades do SI de $kg\cdot m^2s^{-2}$. Esta combinação é também conhecida como \textbf{joule}:

\begin{equation}
    1\ joule=1J=1\ kg\cdot m^2s^{-2}
\end{equation}

\subsubsection{Trabalho}
Quando um objeto se move enquanto um força é exercida sobre ele, então há \textbf{trabalho} a ser feito no objeto pela força.
Se um objeto efetuar um deslocamento $d$ enquanto uma \emph{força constante} \textbf{F} atua nele, a força faz uma quantidade de trabalho igual a

$$
W=F\cdot d=Fdcos\phi
$$

onde $\phi$ é o ângulo entre \textbf{d} e \textbf{F}. O trabalho é também uma grandeza escalar cujas unidades são $N\cdot m$.

O trabalho pode ser negativo; isto acontece quando o ângulo entre a força e o deslocamento é maior que $90^{\circ}$. Também pode ser \emph{zero}; isto acontece se $\phi=90^{\circ}$. Para que exista trabalho, a força tem de ter uma componente ao longo (ou oposta) à direção do deslocamento.
Se várias forças (constantes) atuarem na massa enquanto se move ao longo de um deslocamento \textbf{d}, podemos falar do \textbf{trabalho resultante} feito pelas forças,

$$
\begin{aligned}
    W_{R}&=F_1\cdot d+F_2\cdot d+F_3\cdot d...+F_n\cdot d\\
    &=(\sum F)\cdot d \\
    &=F_{R}\cdot d
\end{aligned}
$$

Se a força que atua sobre o objeto não é constante enquanto o objeto se move, então devemos calcular um integral para encontrar o trabalho realizado.
Supondo que o objeto se mova ao longo de uma linha reta (digamos, ao longo do eixo $x$, de $x_i$ a $x_f$) enquanto
uma força cujo componente $x$ é $F_x(x)$ atua sobre ele. (Ou seja, conhecemos a força $F_x$ como uma função
de $x$.) Então, o trabalho realizado é

\begin{equation}\label{eq:trabalho}
    W=\int_{x_i}^{x_f}F_x(x)dx
\end{equation}

Finalmente, podemos dar a expressão general para o trabalho feito por uma força. Se um objeto se move de $r_i=x_ii+y_ij+z_ik$ a $r_f=x_fi+y_fj+z_fk$ enquanto a força $F(r)$ atua sobre ele, o trabalho feito é:

\begin{equation}
    W=\int_{x_i}^{x_f}F_x(r)dx+\int_{y_i}^{y_f}F_y(r)dy+\int_{z_i}^{z_f}F_z(r)dz
\end{equation}

onde os integrais são calculados ao longo do caminho traçado pelo objeto. Esta expressão pode ser abreviada por

\begin{equation}
W=\int_{r_i}^{r_f}F\cdot dr
\end{equation}

A gravidade funciona em objetos que se movem verticalmente. Pode-se mostrar que se a altura de um objeto mudou numa quantidade $\Delta y$, então a gravidade fez uma quantidade de trabalho igual a 

\begin{equation}
W_{g}=-mg\Delta y
\end{equation}

sem qualquer influência do deslocamento horizontal. Observe-se o sinal de menos aqui; se o objeto aumenta em altura, moveu-se de forma oposta à força da gravidade.

\subsubsection{Força da Mola}
O exemplo mais famoso de uma força cujo valor depende da posição é a força da mola, que descreve a força exercida sobre um objeto até o final de uma \textbf{mola ideal}. Um mola ideal irá puxar o objeto preso à sua extremidade com uma força proporcional à quantidade pela qual é esticada; irá empurrar para fora o objeto anexado a ela com um força proporcional à quantidade de compressão.
Se descrevermos o movimento no final da mola com a coordenada $x$ e colocar-mos a origem do eixo dos $x$ no sítio onde a mola não exerce qualquer força (a dita posição de equilíbrio), então a força da mola é dada por

\begin{equation}
    F_x=-kx
\end{equation}

Aqui, $k$ é um número que é diferente para cada mola ideal e é uma medida que se relaciona com a sua "rigidez" (capacidade de compressão/expansão). A sua unidade é $Nm^{-1}=kgs^{-2}$. Esta equação é conhecida como a \textbf{Lei de Hooke}. Esta lei dá uma descrição decente do comportamento de molas reais, desde que elas possam oscilar sobre suas posições de equilíbrio e não são esticadas demasiado.

O trabalho feito por uma força num objeto anexo ao seu, que se move de $x_i$ até $x_f$, pode ser calculado por

\begin{equation}
    W_{mola}=\frac{1}{2}kx_i^2-\frac{1}{2}kx_f^2
\end{equation}

\subsubsection{O Teorema do Trabalho-Energia Cinética}
Pode-se mostrar que conforme uma partícula se move do ponto $r_i$ para $r_f$, a mudança na energia cinética do objeto é igual ao trabalho resultante feito nele:

\begin{equation}
    \Delta K=K_f-K_i=W_{R}
\end{equation}

\subsubsection{Potência}
Em certas aplicações, estamos interessados na taxa à qual trabalho é realizado por uma força. Se um quantidade de trabalho $W$ é feito num tempo $\Delta t$, então dizemos que a \textbf{potência média} $\bar{P}$ devido à força é

\begin{equation}
    \bar{P}=\frac{W}{\Delta t}
\end{equation}

No limite em que $W$ e $\Delta t$ são muito pequenos, temos a \textbf{potência instantânea} $P$:

\begin{equation}
    P=\frac{dW}{dt}
\end{equation}

A unidade do SI de potência é o \textbf{watt}, definido por:

\begin{equation}
    1\ watt = 1\ W = 1\ jS^{-1}=1\ ks\cdot m^2\cdot s^{-3}
\end{equation}

Pode-se mostrar que se uma força $F$ atua sobre uma partícula que se move com velocidade $v$, então a taxa instantânea na qual o trabalho é feito na partícula é

\begin{equation}
    P=F\cdot v=Fv\cos(\phi)
\end{equation}

onde $\phi$ é o angulo entre as direções de $F$ e $v$.

\subsubsection{Forças Conservativas}
O trabalho feito num objeto pela força da gravidade não depende do caminho percorrido para ir de uma posição a outra. O mesmo é verdade para a força de uma mola. Em ambos os casos, precisamos de saber as coordenadas inicias e finais para computar $W$, o trabalho feito por essa força.
Esta situação também ocorre com a lei geral para a força da gravidade (ver Eq. \ref{eq:lei_da_gravidade}).
Esta situação não se verifica nas forças de atrito vistas anteriormente. As forças de atrito realizam trabalho sobre massas que se movem, mas para calcular o trabalho feito por essas forças, precisamos de saber o \emph{como} as massas foram de um ponto a outro.
Se o trabalho resultante feito por uma força não depender do caminho percorrido entre dois pontos, dizemos que a força é uma \textbf{força conservativa}. Para estas forças, também se verifica que o trabalho resultante feito numa partícula que se move em torno de um caminho fechado é \emph{zero}.

\subsubsection{Energia Potencial}
Para uma força conservativa, é possível encontrar uma função de posição chamada de potencial energia, escrita $U(r)$, da qual podemos encontrar o trabalho realizado pela força.
Supondo que uma partícula se move de $r_i$ a $r_f$. Então, o trabalho feito na partícula por uma força conservativa está relacionado com a correspondente função de energia potencial dada por:

\begin{equation}\label{eq:trabalho-potencial}
    W_{r_i \to r_f}=-\Delta U=U(r_i)-U(r_f)
\end{equation}

A unidade do SI de $U$ é o joules.

Foram encontradas duas forças conservativas até agora. A mais simples é a força da gravidade perto da superfície terrestre, nomeadamente $-mg\Delta y$ para uma massa $m$, onde o eixo $y$ aponta para cima. Para esta força, pode-se mostrar que a energia potencial é

\begin{equation}
    U_{g}=mgy
\end{equation}

Nesta equação, é \emph{arbitrário} onde colocamos a origem do eixo $y$, mas uma vez feita essa escolha, terá de ser mantida.
A outra força conservativa estudada é a força da mola. Uma mola com constante $k$ que é estendida desde a sua posição de equilíbrio por uma quantidade $x$ tem energia potencial dada por

\begin{equation}
    U_{mola}=\frac{1}{2}kx^2
\end{equation}

\subsubsection{Conservação da Energia Mecânica}
Se separarmos as forças do mundo em forças conservativas e não conservativas, então o Teorema do Trabalho-Energia Cinética diz que

\begin{equation}
    W=W_{conservativa}+W_{nao\ conservativa}=\Delta K
\end{equation}

Da Equação \ref{eq:trabalho-potencial}, o trabalho feito por forças \emph{conservativas} pode ser escrito como

$$
W_{conservativas}=-\Delta U
$$

onde $U$ é a soma de \emph{todos} os tipos de energia potencial. Substituindo o resultado acima na Equação \ref{eq:trabalho-potencial}, temos

$$
-\Delta U + W_{n\tilde{a}o\ conservativas}=\Delta K
$$

Reorganizando a equação acima, obtemos o \textbf{teorema geral da Conservação de Energia Mecânica}:

\begin{equation}\label{eq:conservação_em}
    \Delta K+\Delta U=W_{n\tilde{a}o\ conservativas}
\end{equation}

Define-se a \textbf{energia total do sistema} $E$ como a suma da energia cinética e potencial de todos os seus objetos constituintes:

\begin{equation}
    E=K+U
\end{equation}

Então, a Equação \ref{eq:conservação_em} pode ser escrita

\begin{equation}\label{eq:var_energia}
    \Delta E=\Delta K + \Delta U=W_{n\tilde{a}o\ conservativas}
\end{equation}

Por palavras, esta equação diz que a energia mecânica total muda com a quantidade de trabalho feito pelas forças não conservativas.

A maioria dos problemas apresentados são situações onde as forças que atuam nos objetos que se movem são apenas forças conservativas; vagamente falando, isto quer dizer que não há atrito ou que o atrito é negligível.
Se o caso acima de verificar, a Equação \ref{eq:var_energia} pode ser escrita numa forma mais simples:

\begin{equation}
    \Delta E = \Delta K + \Delta U = 0
\end{equation}

Esta equação pode ser escrita:

$$
K_i+U_i=K_f+U_f \qquad ou \qquad E_i=E_f
$$

Noutras palavras, para aqueles casos em que podemos ignorar as forças de atrito, se somarmos todos os tipos de energia para a posição inicial da partícula, é igual à soma de todos os tipos
de energia para a posição final da partícula. Nesse caso, a quantidade de energia mecânica continua o mesmo... é conservada.

A conservação de energia é útil em problemas onde só precisamos de saber as posições ou velocidades, mas não o \emph{tempo} do movimento.

\subsubsection{Trabalho de Forças Não-Conservativas}
Quando, no sistema, atuam forças de atrito, temos de voltar à Equação \ref{eq:var_energia}. A mudança na energia mecânica total é igual ao trabalho feit opelas forças não conservativas:

\begin{equation}
    \Delta E=E_f-E_i=W_{n\tilde{a}o\ conservativas}
\end{equation}

\newpage

\section{Momento Linear e Colisões}
\subsection{Conteúdo Importante}
\subsubsection{Momento Linear}

O \textbf{momento linear} de uma partícula com massa $m$ que se move com uma velocidade $v$ é definido por

\begin{equation}\label{eq:momento_linear}
    p=mv
\end{equation}

O momento linear é um vetor. As unidades do SI para $p$ são $kg\cdot m \cdot s^{-1}$.
O momento de uma partícula está relacionado com a força resultante nessa partícula de uma maneira simples; tendo em conta que a massa de uma partícula permanece constante, se derivar-mos em ordem ao tempo, descobri-mos que

$$
\frac{dp}{dt}=m\frac{dv}{dt}=ma=F_{R}
$$

pelo que

\begin{equation}\label{eq:força-momento}
    F_{R}=\frac{dp}{dt}
\end{equation}

\subsection{Impulso, Força Média}
Quando uma partícula se move livremente, interage com outro sistema por um (breve) período e, depois, move-se livremente novamente, tem uma mudança definida no momento; definimos esta mudança como o impulso $I$ das forças de interação:

\begin{equation}\label{eq:impulso}
    I=p_f-p_i=\Delta p
\end{equation}

O impulso é um vetor e tem as mesmas unidades que o momento linear, $kg\cdot m \cdot s^{-1}$.

Integrando a Equação \ref{eq:força-momento}, podemos mostrar que:

$$
I=\int_{t_i}^{t_f}Fdt=\Delta p
$$

Podemos definir a \textbf{força média} que age sob uma partícula durante um intervalo de tempo $\Delta t$. Esta é:

$$
\bar{F}=\frac{\Delta p}{\Delta t}=\frac{I}{\Delta t}
$$

\subsection{Conservação do Momento Linear}
O momento linear é uma quantidade útil para os casos em que temos algumas partículas (objetos) que interagem entre si, mas não com o resto do mundo. Um sistema desse tipo é chamado um \textbf{sistema isolado}.

Muitas vezes temos motivos para estudar sistemas onde algumas partículas interagem umas com as outras muito rapidamente, com forças que são fortes em comparação com as outras forças do mundo que podem experiênciar. Nessas situações, e por esse breve período de tempo, podemos tratar as partículas como se estivessem isoladas.

Podemos mostrar que quando duas partículas interagem \emph{apenas} consigo mesmas (i.e. estão isoladas) então o seu momento total mantém-se constante:

\begin{equation}\label{eq:conserv-ml}
    p_{1i}+p_{2i}=p_{1f}+p_{2f}
\end{equation}

ou, em termos das suas massas e velocidades,

\begin{equation}
    m_1v_{1i}+m_2v_{2i}=m_1v_{1f}+m_2v_{2f}
\end{equation}

ou, abreviando, $p_1+p_2=P$ (momento total), isto é: $P_i=P_f$.
É importante perceber que a Equação \ref{eq:conserv-ml} é uma equação \emph{vetorial}; diz-nos que o momento total da componente $x$ é conservada e que o total da componente $y$ é, igualmente, conservada.

\subsection{Colisões}
Quando falamos sobre uma colisão na física (entre duas partículas, digamos), queremos dizer que duas partículas movem-se livremente pelo espaço até se aproximarem uma da outra; então, por um
curto período de tempo, elas exercem fortes forças uma sobre a outra até que se separem e movam-se, novamente, livremente.

Para tal evento, as duas partículas têm momentos bem definidos $p_{1i}$ e $p_{2i}$ antes do evento de colisão e $p_{1f}$ e $p_{2f}$ posteriormente. Mas a soma dos momentos antes e depois da colisão é conservada, conforme escrito na Equação \ref{eq:conserv-ml}.

Apesar do momento total ser conservado para um sistema isolado de partículas que colidem, a energia mecânica pode ou não ser conservada. Se a energia mecânica for a mesma antes e depois da colisão, dizemos que a colisão é \textbf{elástica}. Caso contrário, diz-se que a colisão é \textbf{inelástica}.

Se dois objetos colidirem, ficarem juntos e moverem-se como uma massa combinada, diz-se que aconteceu uma \textbf{colisão perfeitamente inelástica}. Pode-se mostrar que em tal colisão mais energia cinética é perdida do que se os objetos ressaltassem um no outro e se afastassem separadamente.

Quando duas partículas sofrem uma colisão \emph{elástica}, sabemos que

$$
\frac{1}{2}m_1v_{1i}^2+\frac{1}{2}m_2v_{2i}^2=\frac{1}{2}m_1v_{1f}^2+\frac{1}{2}m_2v_{2f}^2
$$

No caso especial de uma colisão elástica unidimensional entre as massas $m_1$ e $m_2$, podemos relacionar as velocidades finais às velocidades iniciais. O resultado é

\begin{equation}
    v_{1f}=\frac{m_1-m_2}{m_1+m_2}v_{1i}+\frac{2m_2}{m_1+m_2}v_{2i}
\end{equation}

\begin{equation}
    v_{2f}=\frac{2m_1}{m_1+m_2}v_{1i}+\frac{m_2-m_1}{m_1+m_2}v_{2i}
\end{equation}

Este resultado pode ser útil na resolução de um problema onde ocorre tal colisão, mas não é um equação fundamental. Portanto, não é de útil memorização.

\subsection{Centro de Massa}
Para um sistema de partículas, há um ponto especial no espaço conhecido como o \textbf{centro de massa} que tem uma enorme importância na descrição do movimento do sistema. Este ponto é uma média ponderada das posições de todos os pontos de massa.

Se as particulas de um sistema têm massas $m_1$, $m_2$, ..., $m_N$, com massa total

$$
\sum_{i}^{N}=m_1+m_2+...+m_N\equiv M
$$

e respetivas posições $r_1$, $r_2$, ..., $r_N$, então o centro de massa $r_{CM}$ é

\begin{equation}\label{eq:centro_massa}
    r_{CM}=\frac{1}{M}\sum_{i}^{N}m_ir_i
\end{equation}

o que quer dizer que as coordenadas $x$, $y$ e $z$ do centro de massa são

\begin{equation}\label{eq:centro_massa_coordenadas}
    x_{CM}=\frac{1}{M}\sum_{i}^{N}m_ix_i \qquad y_{CM}=\frac{1}{M}\sum_{i}^{N}m_iy_i \qquad z_{CM}=\frac{1}{M}\sum_{i}^{N}m_iz_i
\end{equation}

Para uma distribuição contínua de massa, a definição de $r_{CM}$ é dado pelo integral sobre os elementos de massa do objeto:

\begin{equation}\label{eq:centro_massa_cont}
    r_{CM}=\frac{1}{M}\int \mathbf{r}dm
\end{equation}

pelo que as coordenadas $x$, $y$ e $z$ do centro de massa são

\begin{equation}\label{eq:centro_massa_cont_coordenadas}
    x_{CM}=\frac{1}{M}\int xdm \qquad y_{CM}=\frac{1}{M}\int ydm \qquad z_{CM}=\frac{1}{M}\int zdm
\end{equation}

Quando as partículas de um sistema estão em movimento, então, em geral, o seu centro de massa está também em movimento. A velocidade do centro de massa é uma semelhante média ponderada das velocidades individuais:

\begin{equation}\label{eq:centro_massa_velocidade}
    v_{CM}=\frac{dr_{CM}}{dt}=\frac{1}{M}\sum_{i}^{N}m_iv_i
\end{equation}

Em geral, o centro de massa vai acelerar; a sua aceleração é dada por

\begin{equation}\label{eq:centro_massa_aceleração}
    a_{CM}=\frac{dv_{CM}}{dt}=\frac{1}{M}\sum_{i}^{N}m_ia_i
\end{equation}

Se $\mathbf{P}$ é o momento total do sistema e $M$ é a massa total do sistema, então o movimento do centro de massa está relacionado com $\mathbf{P}$ por:

$$
v_{CM}=\frac{P}{M} \qquad e \qquad a_{CM}=\frac{1}{M}\frac{dP}{dt}
$$

\subsection{Movimento de um Sistema de Partículas}
Um sistema de muitas partículas (ou um objeto estendido) em geral tem um movimento para o qual o descrição é muito complicada, mas é possível fazer uma declaração simples sobre o
movimento de seu centro de massa. Cada uma das partículas do sistema pode sentir forças de outras partículas do sistema, mas também pode sentir uma força resultante do ambiente (externo); vamos denotar essa força por $F_{ext}$. Descobrimos que quando somamos todas as forças externos agindo sobre todas as partículas de um sistema, dá a aceleração do \emph{centro de massa} de acordo com:

\begin{equation}
    \sum_{i}^{N}F_{ext, i}=Ma_{CM}=\frac{dP}{dt}
\end{equation}

Aqui, $M$ é a massa total do sistema; $F_{ext, i}$ é a força externa que atua na partícula $i$.
Em palavras, podemos expressar o resultado acima da seguinte forma: \emph{para um sistema de partículas, o centro de massa move-se como se fosse uma única partícula de massa $M$ movendo-se sob a influência da soma das forças externas}.

\newpage

\part{Rotações, Vibrações e Ondas}

\newpage

\section{Rotação de um Objeto Sobre um Eixo Fixo}
\subsection{Conteúdo Importante}
\subsubsection{Corpos Rígidos; Rotação}


\newpage

\section{Rolamento; Momento Angular}
\subsection{Conteúdo Importante}
\subsubsection{Rolamento sem Escorregamento}

Quando um corpo rígido simétrico redondo (e.g. uma esfera ou um cilíndro uniforme) de raio $R$ rola sem escorregar sob uma superfície horizontal, a distância percorrida pelo seu centro (quando as suas rodas rodam por um ângulo $\theta$) é igual ao comprimento do arco pelo qual um ponto na extremidade se move:

\begin{equation}
    \Delta x_{CM}=s=R\theta
\end{equation}

\begin{figure}[h!]
    \centering
    \includegraphics[width=0.5\textwidth]{8/fig/rolSemEscor.png}
    \caption{Ilustração da relação entre $\Delta x$, $s$, $R$ e $\theta$ para um objeto em rolamento.}
\end{figure}

A velocidade do centro de massa para o objeto em rolamento, $v_{CM}=\frac{dx_{CM}}{dt}$ e a sua velocidade angular estão relacionados por

\begin{equation}
    v_{CM}=R\omega
\end{equation}

e a magnitude da aceleração do centro de massa está relacionado com a aceleração angular por:

\begin{equation}
    a_{CM}=R\alpha
\end{equation}

A energia cinética do objeto é:

\begin{equation}
    K_{rolamento}=\frac{1}{2}I_{CM}\omega^2+\frac{1}{2}Mv^2_{CM}
\end{equation}

O primeiro termo à direita representa a energia cinética rotacional do objeto sob o seu eixo de simetria; o second termo representa a energia cinética que o objeto teria se se movesse com velocidade $v_{CM}$ sem rolar. No fundo, $K_{rol}=K_{rot}+K_{trans}$.

Quando uma roda rola sem escorregar, \emph{pode} haver força de atrito a agir na superfície da mesma. Neste caso, é uma força de atrito \emph{estático} e dependendo da situação, poderá ter a mesma direção ou oposta ao movimento do centro de massa.

\subsubsection{Torque como Vetor}
No último capítulo, foi dada uma definição para a torque $\tau$ que atua num corpo rígido em rotação sobre um eixo fixo. Agora, é dada uma definição mais geral para "torque"; a "torque" é definida atuando numa única partícula quando uma força atua sobre a mesma.

Supondo que o vetor posição (relativo à origem $O$) de uma partícula é $r$ e uma única força $F$ atua sobre ela. Então, a torque $\tau$ atuando na partícula é

\begin{equation}
    \tau=r\times F
\end{equation}

Se $\phi$ for o ângulo entre o vetor posição $r$ e a força $F$, então a torque $\tau$ tem magnitude

\begin{equation*}
    \tau=rF\sin\phi
\end{equation*}

\subsubsection{Momento Angular de Uma Partícula e de Sistema de Partículas}
Se uma partícula tem um vetor posição $r$ e momento linear $p$, ambos relativos a uma origem $O$, então o \textbf{momento angular} dessa partícula (relativo à origem) é definido por:

\begin{equation}
    L=r\times p=m(r\times v)
\end{equation}

O momento angular tem unidades $kg\cdot m^2\cdot s^{-1}$.
É possível mostrar que a torque resultante numa partícula é igual à derivada em ordem ao tempo do momento angular:

\begin{equation}
    \sum \tau=\frac{dL}{dt}
\end{equation}

Para um conjunto de pontos de massa em movimento, o momento angular total é definido como a soma vetorial dos momentos angulares individuais:

\begin{equation*}
    L=L_1+L_2+L_3+...
\end{equation*}

A taxa à qual o momento angular total muda está relacionado com as torques que surgem de forças exceridas fora do sistema.

\begin{equation}
    \sum \tau_{ext}=\frac{dL}{dt}
\end{equation}

Isto diz-nos que quandoi a soma das torques externas é zero, então \textbf{L} é constante (conservado).

\subsubsection{Momento Angular para Rotação sobre Eixo Fixo}
Para uma rotação sobre um eixo fixo, o "momento angular" do objeto rígido é um número, $L$. Além disso, é possível mostrar que se a avelocidade angular do objeto for $\omega$ e o seu momento de inércia sobre o eixo dado é $I$, então o seu momento angular sobre o eixo é

\begin{equation}
    L=I\omega
\end{equation}

\subsubsection{Conservação do Momento Angular}
Para um sistema no qual não há torque externa resultante, o momento angular total mantém-se constante: $L_i=L_f$. Este princípio é conhecido como a \textbf{Conservação do Momento Angular}.

\newpage

\section{Equilíbrio Estático}
\subsection{Conteúdo Importante}

Neste capítulo, estudamos um caso especial da dinâmica de objetos rígidos abrangidos nos últimos dois capítulos. É o caso especial (muito importante!) pnde o centro de massa do objeto não tem movimento e o objeto não está em rotação.

\subsubsection{Condições para Equilíbrio de um Corpo Rígido}
Para um objeto rígido que não se move, temos as seguintes condições:

\begin{itemize}
    \item a soma (vetorial) das forças externas no corpo rígido deve ser zero:
    \begin{equation}
        \sum F =0
    \end{equation}
    Quando esta condição é satisfeita, diz-se que o objeto está em \textbf{equilíbrio translacional}.
    \item a soma das torques externas no corpo rígido deve ser zero:
    \begin{equation}
        \sum \tau = 0
    \end{equation}
    Quando esta condição é satisfeita, dizemos que o objeto está em \textbf{equilíbrio rotacional}.
\end{itemize}

Quando ambas as condições acimas são satisfeitas, diz-se que o corpo está em \textbf{equilíbrio estático}.

\subsubsection{Exemplos de Corpos Rígidos em Equilíbrio Estático}
Estratégia para resolver problemas em equilíbrio estático:
\begin{enumerate}
    \item determinar todas as forças que atuam no corpo rígido. Essas forças virão de outros objetos com o qual o corpo está em contacto (suportes, paredes, chão, massas em repouso sob ele) bem como a gravidade;
    \item desenhar um diagrama e colocar toda a informação existente sobre essas forças: os pontos do corpo no qual elas agem, as magnitudes, as direções;
    \item escrever as equações para o equilíbrio estático. Para a equação da torque, vhá opção de escolha de onde colocar o eixo; ao fazer essa escolha, há que pensar em qual ponto tornaria as equações resultantes mais simples;
    \item resolver as equações.
\end{enumerate}

\newpage

\section{Movimento Oscilatório}
\subsection{Conteúdo Importante}
\subsubsection{Movimento Harmónico Simples}

Neste capítulo, consideramos sistemas que têm um movimento que se repete ao longo do tempo, isto é, é um movimento \textbf{periódico}. Em particular, consideramos sistemas que têm uma coordenada (e.g. $x$) que tem uma dependência sinusoidal com o tempo.

Um gráfico de $x\ vs.\ t$ para este tipo de movimento é mostrado na figura abaixo. Supondo que a partícula tem uma movimento sinusoidal periódico no eixo dos $x$, e o seu movimento oscila entre $x=+A$ e $x=-A$. Então, a expressão geral de $x(t)$ é dada por

\begin{equation}\label{eq:harmonic}
    x(t)=Acos(\omega t+\phi)
\end{equation}

$A$ é a \textbf{amplitude} do movimento. $\omega$ é a \textbf{frequência angular}. 

Da Equação \ref{eq:harmonic}, infere-se que quando o tempo $t$ aumenta em  $\frac{2\pi}{\omega}$, o argumento do $\cos$ aumenta em $2\pi$ e o valor para $x$ vai ser o mesmo. Portanto, o movimento repete-se após um intervalo de tempo $\frac{2\pi}{\omega}$, denotado por $T$, como o período do movimento.

\begin{figure}[h!]
    \centering
    \includegraphics[width=0.5\textwidth]{10/fig/harmonic.png}
    \caption{Gráfico posição-tempo para movimento harmónico simples.}
\end{figure}

O número de oscilações por tempo é dado por $f=\frac{1}{T}$, chamada a \textbf{frequência} do movimento:

\begin{equation}
    T=\frac{2\pi}{\omega} \qquad f=\frac{1}{T}=\frac{\omega}{2\pi}
\end{equation}

Reagrupando os termos, temos uma fórmula para $\omega$ em termos de $f$ ou $T$:

\begin{equation}
    \omega=2\pi f=\frac{2\pi}{T}
\end{equation}

De $x(t)$, podemos obter a velocidade e a aceleração da partícula:

\begin{equation}
    v(t)=\frac{dx}{dt}=-\omega A\sin(\omega t+\phi)
\end{equation}

\begin{equation}\label{eq:harm-acc}
    a(t)=\frac{dv}{dt}=-\omega^2A\cos(\omega t + \phi)
\end{equation}

Note-se que os valores máximos de $v$ e $a$ são:

\begin{equation}
    v_{max}=\omega A \qquad a_{max}=\omega^2A
\end{equation}

A partícula atinge $v_{max}$ no \emph{meio} da oscilação (quando $x=0$). A magnitude da aceleração é maior nos extremos da oscilação (quando $x=\pm A$).

Comparando a Equação \ref{eq:harmonic} e a Equação \ref{eq:harm-acc}, temos:

\begin{equation}\label{eq:osc-massa-pos-1}
    \frac{d^2x}{dt^2}=-\omega^2x
\end{equation}

que é igual a $a(t)=-\omega^2x(t)$. É possível mostrar que a seguinte relação entre a velocidade $|v(t)|$ e a coordenada $x(t)$:

\begin{equation}
    |v(t)|=\omega A|\sin(\omega t+\phi)|=
    \omega A \sqrt{1-(\cos(\omega t + \phi)^2)}=\omega A\sqrt{1-(\frac{x(t)}{A})^2}.
\end{equation}

\subsubsection{Massa Ligada a uma Mola}
Supondo que uma massa $m$ está ligada à extremidade de uma mola com constante de força $k$ e desliza sobre uma superfície sem atrito.

\begin{figure}[h!]
    \centering
    \includegraphics[width=0.5\textwidth]{10/fig/sistema.png}
    \caption{Massa $m$ ligada a uma mola horizontal com constande de força $k$; $m$ desliza sobre uma superfície sem atrito.}
\end{figure}

Então, se medirmos a coordenada $x$ da massa desde o ponto onde essa mesma massa estaria se a mola estivesse em equilíbrio, a Segunda Lei de Newton diz que:

\begin{equation*}\label{eq:osc-pos-massa-2}
    F_x=-kx=ma_x=m\frac{d^2x}{dt^2}
\end{equation*}

tendo então

\begin{equation}
    \frac{d^2x}{dt^2}=-\frac{k}{m}x
\end{equation}

Comparando a Equação \ref{eq:osc-pos-massa-2} e \ref{eq:osc-pos-massa-1}, podemos identificar $\omega^2$ como $\frac{k}{m}$ para que se tenha

\begin{equation}
    \omega=\sqrt{\frac{k}{m}}
\end{equation}

Da frequência angular $\omega$ podemos descobrir o tempo $T$ e a frequência $f$ do movimento:

\begin{equation}
    T=\frac{2\pi}{\omega}=2\pi\sqrt{\frac{m}{k}} \qquad f=\frac{1}{T}=\frac{1}{2\pi}\sqrt{\frac{k}{m}}
\end{equation}

Atente-se que $\omega$ (e, por consequência, $T$ e $f$) não dependem da amplitude $A$ do movimento da massa. Na realidade, se o movimento da massa for muito grande, então a mola não obedecerá a Lei de Hooke, mas desde que as oscilações sejam "pequenas", o periodo é o mesmo para todas as amplitudes.

\subsubsection{Energia e Oscilador Harmónico Simples}

Para o sistema massa-mola, a energia cinética é dada por

\begin{equation}\label{eq:osc-k}
    K=\frac{1}{2}mv^2=\frac{1}{2}m\omega^2A^2(\sin(\omega t + \phi)^2)
\end{equation}

e a energia potencial é

\begin{equation}
    U=\frac{1}{2}kx^2=\frac{1}{2}kA^2(\cos(\omega t +\phi)^2)
\end{equation}

Usando $\omega^2=\frac{k}{m}$ na Equação \ref{eq:osc-k}, temos que a energia total do sistema é:

\begin{equation}
    E=K+U=\frac{1}{2}kA^2[(\sin(\omega t + \phi))^2+(\cos(\omega t + \phi)^2)]
\end{equation}

Da entidade trigonométrica $\sin^2\theta+\cos^2\theta=1$, temos

\begin{equation}
    E=\frac{1}{2}kA^2
\end{equation}

mostrando que a energia de um oscilador harmónico simples (como exemplificado pelo sistema massa-mola) é constante e igual à energia potencial da mola quando está estendida ao máximo (no instante de tempo em que a massa não tem movimento).

Da Equação \ref{eq:osc-k}, temos a energia cinética em função do tempo:

\begin{equation*}
    K=\frac{1}{2}mv^2=\frac{1}{2}m\omega^2A^2\sin^2(\omega t +\phi)
\end{equation*}

O valor máximo da energia cinética é $\frac{1}{2}m\omega^2A^2$, que ocorre quando $x=0$. Visto que podemos decidir o "ponto zero" da energia potencial, temos que $U(x)=0$ em $x=0$. Então, a energia total do sistema é igual ao valor máximo da energia cinética:

\begin{equation*}
    E=\frac{1}{2}m\omega^2A^2
\end{equation*}

Usando as expressões acima, temos a energia potencial do sistema:

$$
\begin{aligned}
    U&=E-K \\
    &=\frac{1}{2}m\omega^2A^2-\frac{1}{2}m\omega^2A^2\sin^2(\omega t +\phi)=\frac{1}{2}m\omega^2A^2(1-\sin^2(\omega t + \phi)) \\
    &=\frac{1}{2}m\omega^2A^2\cos^2(\omega t + \phi) \\
    &=\frac{1}{2}m\omega^2x^2
\end{aligned}
$$

Para o sistema massa-mola, $U$ é dada por $\frac{1}{2}kx^2$, que nos dá a relação $m\omega^2=k$ ou $\omega=\sqrt{\frac{k}{m}}$, encontrada anteriormente. Usando a relação $v_{max}=\omega A$, a energia potencial pode ser escrita como

\begin{equation}
    U(x)=\frac{1}{2}m\omega^2x^2=\frac{1}{2}\frac{mv^2_{max}}{A^2}x^2
\end{equation}

\subsubsection{Relação com o Movimento Circular Uniforme}

Existe uma correspondência entre o movimento harmónico simples e o movimento circular uniforme, ilustrado na figura abaixo.

\begin{figure}[h!]
    \centering
    \includegraphics[width=0.5\textwidth]{10/fig/mhs-mcu.png}
    \caption{Massa $m$ ligada a uma mola horizontal com constande de força $k$; $m$ desliza sobre uma superfície sem atrito.}
\end{figure}

Na figura (a), uma massa move-se num caminho circular horizontal com movimento circular uniforme de raio $R$. A sua velocidade angular é $\omega$, logo a sua localização é dada por

\begin{equation*}
    \theta(t)=\omega t + \phi
\end{equation*}

Na figura (b) está representado o movimento da massa como se fosse vista por alguém que estivesse na direção $+y$ ao nível do disco onde a massa roda. Tal observador apenas vê mudança na coordenada $x$. Visto que $x=Rcos\theta$, a coordenada observada é

\begin{equation*}
    x(t)=Rcos(\theta(t))=Rcos(\omega t + \phi)
\end{equation*}

\subsubsection{Pêndulo}

\begin{figure}[h!]
    \centering
    \includegraphics[width=0.5\textwidth]{10/fig/pendulos.png}
    \caption{(a) Pêndulo simples. (b) Pêndulo físico.}
\end{figure}

Abordamos, primeiro, o \textbf{pêndulo simples}, que tem uma massa $m$ suspensa por uma corda de comprimento $L$ cuja massa pode ser ignorada. A massa é colocada em movimento num plano vertical. É possível mostrar que se $\theta$ é o ângulo que a corda faz com a vertical, obedece à seguinte equação:

\begin{equation}
    \frac{d^2\theta}{dt^2}=-\frac{g}{L}\sin\theta
\end{equation}

Quando se restringe $\theta$ a valores "pequenos", podemos usar a aproximação $\sin \theta \approx \theta$, que é verdade se $\theta$ for medido em radianos. Nesse caso, a equação acima é reescrita como

\begin{equation}
    \frac{d^2\theta}{dt^2}=-\frac{g}{L}\theta
\end{equation}

Comparando a equação acima com a Equação \ref{eq:osc-massa-pos-1}, podemos identificar a frequência angular do movimento:

\begin{equation}
    \omega=\sqrt{\frac{g}{L}}
\end{equation}

\begin{equation}
    T=\frac{2\pi}{\omega} \qquad f=\frac{1}{T}=\frac{1}{2\pi}\sqrt{\frac{g}{L}}
\end{equation}

Das equações acima, infere-se que não têm qualquer dependência com a massa suspensa na corda ou com a amplitude do balanço, desde que o ângulo $\theta$ seja pequeno.

\begin{equation}
    \theta(t)=\theta_{max}\cos(\omega t + \phi)
\end{equation}

Uma generalização do pêndulo simples é a de um corpo rígido que está livre para girar num plano em torno de um eixo sem atrito. Tal sistema é conhecido como um \textbf{pêndulo físico}.

Supondo que olhamos em direção a linha uma que une o eixo ao centro de massa do objeto. Se $\theta$ for o ângulo que esta linha faz com a vertical, e se de novo for usada a aproximação $\sin\theta \approx \theta$, é possível mostrar que obedece à seguinte equação

\begin{equation}
    \frac{d^2\theta}{dt^2}=-\frac{Mgd}{I}\theta
\end{equation}

onde $d$ é a distância entre o eixo e oc entro de massa, $M$ é a massa do objeto e $I$ é o momento de inércia do objeto sobre o eixo dado.

O período $T$ é dado por:

\begin{equation}
    T=\frac{2\pi}{\omega}=2\pi\sqrt{\frac{I}{Mgd}}
\end{equation}

\newpage

\section{Ondas I: Generalizações, Sobreposição e Ondas Estacionárias}
\subsection{Conteúdo Importante}
\subsubsection{Movimento Ondulatório}

O \textbf{movimento ondulatório} ocorre quando os elementos de massa deu um meio, tal como uma corda esticada ou a superfície de um líquido, faz movimentos oscilatórios relativamente pequenos, mas que, coletivamente, fornecem um padrão que viaja por longas distâncias. Este tipo de movimento também inclui o fenómeno do \emph{som}, onde as moléculas do ar à nossa volta fazem pequenas oscilações mas, coletivamente, causam uma perturbação que pode viajar uma longa distância.

\subsubsection{Tipos de Ondas}
Em alguns tipos de movimento ondulatório, o movimento dos elementos do meio é (para a maior parte) perpendicular ao movimento da perturbação. Isto é verdade para ondas numa corda, por exemplo. Este tipo de onda é chamada \textbf{onda transversal} e é o tipo mais fácil de visualizar.

\begin{figure}[h!]
    \centering
    \includegraphics[width=0.5\textwidth]{11/fig/Transverse-Waves.png}
    \caption{Onda transversal.}
\end{figure}

Para outras ondas, o movimento dos elementos do meio é paralelo ao movimento da perturbação. Este tipo de onda pode ser vista quando alongamos uma mola e abanamos a sua extremidade paralelamente ao seu comprimento. Observar-se, posteriormente, regiões onde a mola está mais comprimida e esticada. Uma onda sonora viaja da mesma maneira; neste caso, as moléculas de ar têm um pequeno movimento em forma de \emph{"vai-e-volta"} que egra regiões onde o ar tem maior e menor compressão.

Uma onda viaja ao longo da direção de propagação é uma \textbf{onda longitudinal}.

\begin{figure}[h!]
    \centering
    \includegraphics[width=0.5\textwidth]{11/fig/Longitudinal-Waves-1.png}
    \caption{Onda longitudinal.}
\end{figure}

\subsubsection{Descrição Matemática de uma Onda; Comprimento de Onda, Frequência e Velocidade da Onda}

No tratamento matemático dos fenómenos ondulatórios, lidaremos maioritariamente com ondas que viajam em uma dimensão. A coordenada ao longo da qual a perturbação viaja será $x$; em cada valor de $x$, o meio irá se "deslocar" de alguma maneira, e esse deslocamento será descrito pela variável $y$. Então, $y$ dependerá de $x$ e do tempo $t$ no qual se vê o deslocamento. Em geral, $y=f(x,t)$.

Se nos especializarmos no caso em que a forma da onda não muda com o tempo, mas em vez disso viaja ao longo do eixo $+x$ com uma velocidade $v$, então a onda será apenas uma função da combinação $x-vt$:

\begin{equation*}
    y=f(x-vt) \qquad \text{(Velocidade v na direção +x).}
\end{equation*}

A partir da expressão acima, segue-se que uma onda viajando na outra direção é:

\begin{equation*}
    y=f(x+vt) \qquad \text{(Velocidade v na direção -x).}
\end{equation*}

Por razões que são basicamente matemáticas, é importante estudar uma onda particular que tenha uma forma sinoidal em função de $x$. Essa onda é dada por

\begin{equation}\label{eq:onda-sin}
    y(x,t)=y_m\sin(kx\mp \omega t)
\end{equation}

Nesta equação, o sinal $-$ é utilizado para uma onda que viaje na direção $+x$; $+$ é escolhido para uma onda que viaja na direção $-x$.

Esta forma de onda representa um \emph{comboio de ondas} (infinito) em vez de um pulso. Na Equação \ref{eq:onda-sin}, $k$ é o \textbf{número de onda angular}, cuja unidade é $m^{-1}$. $\omega$ é a \textbf{frequência angular} e tem unidades $s^-1$.

A onda harmónica na Equação \ref{eq:onda-sin} é periódica no espaço e no tempo. O \textbf{comprimento de onda} $\lambda$ da onda é a distância entre repetições da forma sinusoidal da onda quando "congelamos" a onda no tempo. É possível mostrar que esse comprimento está relacionado com $k$ por:

\begin{equation}
    k=\frac{2\pi}{\lambda}
\end{equation}

O \textbf{período} $T$ da onda é o tempo entre repetições do movimento de qualquer elemento do meio. É possível mostrar que está relacionado com a frequência angular por:

\begin{equation}
    \omega=\frac{2\pi}{T}
\end{equation}

Tal como no estudo das oscilações, a \textbf{frequência} da onda tem a mesma relação com $\omega$ e $T$:

\begin{equation*}
    f=\frac{1}{T}=\frac{\omega}{2\pi}
\end{equation*}

É possível mostrar que a velocidade de tal onda (i.e. a taxa a qye a sya crusta viaja ao longo do eixo $x$) é:

\begin{equation}
    v=\frac{\omega}{k}=\lambda f
\end{equation}

Cada ponto de massa do meio move-se como uma oscilador harmónico cuja amplitude e frequência são iguais às da onda; a velocidade máxima de cada elemento é $v_{max}=y_m\omega$ e aceleração máxima de cada elemento é $y_m\omega^2$.

\subsubsection{Ondas numa Corda Esticada}
Um dos exemplos mais simples de uma onda transversal é o de ondas que viajam muma corda esticada. Pode-se mostrar que para uma corda sob uma tensão $\tau$, cuja massa por unidade de comprimento é dada por $\mu$, a velocidade das ondas é

\begin{equation}
    v=\sqrt{\frac{\tau}{\mu}}
\end{equation}

Uma caraterística importante das ondas é que \emph{transmitem energia}. A \textbf{potência média} é a taxa À qual energia mecânica passa qualquer ponto do eixo $x$. É possível mostrar que para ondas harmónicas numa corda, esta quantidade é dada por

\begin{equation}
    \bar{P}=\frac{1}{2}\mu v\omega^2y_m^2
\end{equation}

onde $\mu$ é densidade linear da corda, $v$ é a velocidade da onda e $\omega$ é a frequência angular da onda harmónica.

\subsubsection{Princípio da Sobreposição}
Uma propriedade simples de todas as ondas da natureza que estudamos é que elas se somam. Mais precisamente, se uma perturbação física de um meio gerar a onda $y_1(x,t)$ e outra perturbação gera a onda $y_2(x,t)$ então se ambos os efeitos agirem ao mesmo tempo, a onda resultante será

\begin{equation}
    y_{Tot}(x,t)=y_1(x,t)+y_2(x,t)
\end{equation}

\subsubsection{Interferência de Ondas}
Nesta e na próxima secção, são dados resultados para a sobreposição de duas ondas harmónicas que apenas diferem num aspeto.

Primeiro, consideramos duas ondas com a mesma velocidade, frequência, amplitude e direção de movimento mas que diferem por uma constante de fase. Combinaremos as duas ondas

\begin{equation}
    y_1=y_m\sin(kx-\omega t) \qquad y_2=y_m\sin(kx-\omega t + \phi)
\end{equation}

Para chegar a uma forma útil para a soma dessas duas ondas, pode-se usar a identidade trigonométrica

\begin{equation*}
    \sin\alpha+\sin\beta=2\sin\frac{1}{2}(\alpha+\beta)\cos\frac{1}{2}(\alpha-\beta)
\end{equation*}

Com isto, podemos mostrar que a onda resultante $y'(x,t)$ é:

\begin{equation}
    y'(x,t)=y_1(x,t)+y_2(x,t)=[2y_m\cos(\frac{1}{2}\phi)]\sin(kx-\omega t + \frac{1}{2}\phi)
\end{equation}

A onda resultante tem uma nova fase, mais importante, tem uma nova amplitude que depende de $y_m$ e $\phi$:

\begin{equation}
    y'_m=2y_m\cos\frac{1}{2}\phi
\end{equation}

Quando $\phi=0$, a nova amplitude é $2y_m$; as ondas dizem-se estar \textbf{completamente em fase} e que a adição é \textbf{completamente construtiva}. Um máximo da onda $y_1$ coincide com um máximo da onda $y_2$ e uma onda "maior" é o resultado.
Quando $\phi=\pi$, a nova amplitude é \emph{zero} e as ondas dizem-se estar \textbf{completamente fora de fase} e que a adição é \textbf{completamente destrutiva}. Neste caso, um máximo da onda $y_1$ coincide com um mínimo da onda $y_2$ e o resultado é o cancelamento completo.

\subsubsection{Ondas Estacionárias}
Neste capítulo, é considerado o resultado da adição de duas ondas harmónicas que têm a mesma velocidade, frequência e amplitude mas para quais as direções de propagação ($+x$ ou $-x$ são diferentes). Neste caso, as constantes de fase para cada onda são irrelevantes. Serão adicionadas as ondas:

\begin{equation}
    y_1=y_m\sin(kx-\omega t) \qquad y_2=y_m\sin(kx+\omega t)
\end{equation}

Novamente, podemos usar a identidade trigonométrica para a adição de dois $\sin$ e mostrar que a soma $y'_m(x,t)$ é:

\begin{equation}\label{eq:onda-estac}
    y'_(x,t)=y_1(x,t)+y_2(x,t)=[2y_m\cos(\omega t)]\sin(kx)
\end{equation}

A onda resultante da Equação \ref{eq:onda-estac} é uma função interessante de $x$ e $t$; no entnato não é uma onda que se propaga por não ser da forma $f(kx\mp\omega t)$. É uma função sinusoidal da coordenada $x$, multiplicada por um factor modulador $\cos(\omega t)$. Uma onda na forma da Equação \ref{eq:onda-estac} é uma \textbf{onda estacionária}.

Para a onda na Equação \ref{eq:onda-estac}, existem pontos onde não há deslocamento, i.e. aqueles onde $\sin(kx)=0$ (em que $x=\frac{n\pi}{k}$, $n$ natural). Estes pontos são os \textbf{nós} do padrão da onda estacionária.

Existem pontos para o qual o deslocamento é o máximo, nomeadamente aqueles para o qual $\sin(kx)=\pm1$ (onde $x=\frac{(2n+1)\pi}{2k}$, $n$ natural). Estes pontos são chamados os \textbf{antinós} do padrão da onda estacionária. Nós e antinós consecutivos estão separados por $\frac{\lambda}{2}$, onde $\lambda$ é o comprimento de onda das ondas originais que formaram a onda estacionária. Um nó e o antinó mais próximo estão separados por $\frac{\lambda}{4}$.

\subsubsection{Ondas Estacionárias em Cordas Sob Tensão}
Os modos de oscilação da corda são aqueles em que a corda oscila com nós em uma das extremidades e um padrão especial de nós e antinós entre eles. Esse padrão apenas existe quando a corda vibra com certas \textbf{frequências resonantes}. Para esses nós, o comprimento da corda $L$ é um múltiplo de $\frac{\lambda}{2}$:

\begin{equation*}
    L=n\frac{\lambda}{2} \qquad n=1,2,3...
\end{equation*}

que leva à fórmula das frequências resonantes:

\begin{equation}
    f_n=n\frac{v}{2L} \qquad n=1,2,3...
\end{equation}

onde $v$ é a velocidade das ondas na corda e $f_n$ é a frequência resonante do $n$-ésimo nó.

\end{document}
