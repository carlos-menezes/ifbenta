\subsection{Segunda Lei de Newton}
\textbf{1.} Uma massa de 3.0kg sofre uma aceleração dada por $a=(2.0i + 5.0j)ms^{-2}$. Qual é a força resultante e a sua magnitude?
\linebreak
A Segunda Lei de Newton diz-se que a força resultante numa massa $m$ é $\sum F = ma$. Portanto,

$$
\begin{aligned}
    F_{R}&=ma \\
        &=(3.0\ kg)(2.0i+5.0j)ms^{-2} \\
        &=(6.0i + 15j)N
\end{aligned}
$$

A magnitude da $F_{R}$ é dada pelo cálculo da norma do vetor $F_{R}$:

$$
\begin{aligned}
    F_{R}=\sqrt{(6.0N)^2+(15.0N)^2}=16N
\end{aligned}
$$

\textbf{2.} Enquanto duas forças agem sobre ela, uma massa de $m=3.2kg$ move-se com uma velocidade constante $(3ms)i-(4ms)j$. Uma dessas forças é $F_1=(2N)i+(-6N)j$. Qual é a outra força?
\linebreak
A Segunda Lei de Newton diz-se que a força resultante numa massa $m$ é $\sum F = ma$. Portanto, $\sum F = ma \equiv F_1+F_2=ma$. Neste caso, a velocidade é constante, pelo que $a=0$. Portanto,

$$
\begin{aligned}
    F_1+F_2&=0 \\
    \implies F_2&=-F_1 \\
    \implies F_2&=-[(2N)i+(-6N)j] \\
    \implies F_2&=(-2N)i+(6N)j
\end{aligned}
$$

\textbf{3.} Um objeto de massa $m=4.0\ kg$ tem uma velocidade de $3.0i\ ms^{-1}$ num dado instante. Oito segundos depois, a sua velocidade é $(8.0i+10.0j)ms^{-1}$. Assumindo que o objeto foi sujeito a uma força resultante constante, encontre (a) as componentes da força e (b) a sua magnitude. \\
\linebreak
\textbf{(a)} É-nos dito que a força resultante que age na massa foi constante. Logo, sabemos que a sua aceleração foi igualmente constante, we podemos usar resultados de capítulos anteriores. São facultadas as velocidades inicias e final, pelo que é possível calcular as componentes da aceleração:

$$
a_x=\frac{\Delta v_x}{\Delta t}=\frac{[(8.0\ ms^{-1})-(3.0\ ms^{-1})]}{(8.0\ s)}=0.63\ ms^{-2}
$$

$$
a_y=\frac{\Delta v_y}{\Delta t}=\frac{[(1.0\ ms^{-1})-(0.0\ ms^{-1})]}{(8.0\ s)}=1.3\ ms^{-2}
$$

Visto que a massa é facultada no enunciado, é possível calcular as componentes da força resultante:

$$
\begin{aligned}
    F_x=ma_x=(4.0\ kg)(0.63\ ms^{-2})=2.5N \\
    F_y=ma_y=(4.0\ kg)(1.3\ ms^{-2})=5.0N \\
\end{aligned}
$$

\textbf{(b)} A magnitude da força resultante é obtida através da seguinte computação:

$$
F=\sqrt{F_x^2+F_y^2}=\sqrt{(2.5\ N)^2+(5.0\ N)^2}=5.6\ N
$$

A direção $\theta$ da força $F$ é dada por

$$
\tan \theta = \frac{F_y}{F_x} = 2.0 \quad \implies \quad \theta = \arctan(2.0)=63.4^{\circ}
$$

\textbf{4.} Cinco forças atuam sobre a caixa de massa $4.0\ kg$ na Figura \ref{fig:4caixa}. Encontre a aceleração da caixa (a) na notação unidade-vetor e (b) a sua magnitude e direção.
\linebreak
\begin{figure}[h!]
    \centering
    \includegraphics[width=0.5\textwidth]{forças/fig/ex4.png}
    \caption{Cinco forças atuam num caixa de massa $4.0\ kg$.}
    \label{fig:4caixa}
\end{figure}


\textbf{(a)} Existe a necessidade de separar as forças nas suas duas componentes $x$ e $y$. A Segunda Lei de Newton será utilizada para computar o valor da aceleração da caixa.

$$
\begin{aligned}
    \sum F_x&=-11N+3N+14\cos(30º) \\
    &=4.1N
\end{aligned}
$$

$$
\begin{aligned}
    \sum F_y&=+5N-17N+14\sin(30) \\
    &=-5.0N
\end{aligned}
$$

Logo, tem-se que a força resultante é

$$
\sum F=(4.1N)i+(-5.0N)j
$$

Utilizando a Equação \ref{eqn:2newton}, temos:

$$
    \sum a_x=\frac{\sum F_x}{m}=\frac{(4.1\ N)}{4.0\ kg}=1.0\ ms^{-2} \\
    \sum a_y=\frac{\sum F_y}{m}=\frac{(-5.0\ N)}{4.0\ kg}=-1.2\ ms^{-2} \\
$$


O valor da aceleração em notação vetorial é, então,

$$
a=(1.0i-1.2j)ms^{-2}
$$

\textbf{(b)} A aceleração encontrada em (a) tem magnitude

$$
a=\sqrt{a_x^2+x_y^2}=\sqrt{(4.0\ ms^{-2})^2+(-1.2\ ms^{-2})^2}=1.6\ ms^{-2}
$$

A direção $\theta$ do vetor aceleração é dada por

$$
\tan \theta = \frac{a_y}{a_x} = -1.2 \quad \implies \quad \theta = \arctan(-1.2)=-50^{\circ}
$$

Neste caso, tendo em conta que $a_y$ é negativo e $a_x$ é positivo, a escolha de $\theta = -50^{\circ}$ está correta.

\subsection{Exemplos de Forças}
\textbf{5.} Encontre a tensão em cada uma das cordas da Figura \ref{fig:5caixa}.
\linebreak
\begin{figure}[h!]
    \centering
    \includegraphics[width=0.5\textwidth]{forças/fig/ex5.png}
    \caption{Massas suspensas por cordas.}
    \label{fig:5caixa}
\end{figure}

\textbf{(a)} Seja $m_1$ a massa correspondente à bola representada na Figura \ref{fig:5caixa} (a). A força gravítica atua para baixo com uma força de magnitude $m_1g$. A corda vertical puxa "para cima" com uma força de magnitude $T_3$. Tendo em conta que a massa pendurada não tem aceleração, verifica-se que $T_3=m_1g$. Portanto, o valor de $T_3$ é dado por:

$$
T_3=m_1g=(5.0\ kg)(9.9\ ms^{-2})=49\ N
$$

Considerando agora o ponto de união das três cordas. Esse ponto não tem qualquer aceleração, pelo que o resultado da força resultante deverá ser zero. As componentes verticais e horizontains dessas forças somam a zero, separadamente.

$$
    \begin{cases}
        -T_1\cos(40^{\circ})+T_2\cos(50^{\circ})=0 \\
        T_1\sin(40^{\circ})+T_2\sin(50^{\circ})-T_3=0
    \end{cases}
$$

A primeira equação, a soma das componentes horizontais, dá-nos $T_2=1.19T_1$.

A segunda equação,  a soma das componentes verticais, dá-nos o valor  $T_1=31.5\ N$.

Sabendo o valor de $T_1$, $T_2=1.19T_1\ \implies \ T_2=37.5\ N$

As tensões no sistema (a) são, portanto:

$$
T_1=31.5\ N \qquad T_2=37.5\ N \qquad T_3=49\ N
$$

\textbf{(b)} A força resultante na massa pendurada, $m_2$, tem de ser 0, porque não há aceleração. Visto que a gravida "puxa para baixo" com uma força $m_2g$ e a corda vertical "puxa para cima" com uma força $T_3$, sabemos que

$$
T_3-F_g=0 \implies T_3=F_g \implies T_3=m_2g \\
\begin{aligned}
    T_3&=m_2g
    &=(10\ kg)\cdot(9.8\ ms^{-2})=98\ N
\end{aligned}
$$

Considerando agora as forças que atuam no ponto onde todas as forças se encontram. Novamente, como não há aceleração nesse ponto, tanto as componentes verticais como horizontais somam a zero.

No que toca às forças horizontais, temos:

$$
-T_1\cos(60^{\circ})+T_2=0 \qquad \implies \qquad T_2=T_1\cos(60^{\circ}) \\
$$

A soma das forças verticais é dada por:

$$
T_1\sin(60^{\circ})-T_3=0 \qquad \implies \qquad T_1=\frac{T_3}{\sin(60^{\circ})}=113\ N
$$

Podemos, depois, obter $T_2$:

$$
T_2=T_1\cos(60^{\circ})=(133\ N)\cos(60^{\circ})=56.6\ N
$$

As tensões no sistema (b) são, portanto:

$$
T_1=133\ N \qquad T_2=56.6\ N \qquad T_3=98\ N
$$

\textbf{6.} Um bloco de massa $m=2.0\ kg$ está suspenso em equilíbrio num encosta que faz um ângulo $\theta=60^{\circ}$ pela força horizontal $F$, como mostrado na Figura \ref{fig:6plano}. (a) Determine o valor de $F$, a magnitude de $F$. (b) Determine a força normal exercida pelo plano inclinado no bloco (ignore o atrito).
\linebreak
\begin{figure}[h!]
    \centering
    \includegraphics[width=0.5\textwidth]{forças/fig/ex6.png}
    \caption{(a) Bloco em repouso numa rampa sem atrito por uma força horizontal. (b) Forças a atuarem no bloco.)}
    \label{fig:6plano}
\end{figure}

\textbf{(a)} Fazer um diagrama de das forças que atuam no corpo torna-se essencial para este problema, daí a inclusão da \ref{fig:6plano} (b). Muitas vezes, para problemas envolvendo um bloco num plano inclinado, é mais fácil usar as componentes da $F_g$ ao longo desse mesmo plano inclinado e perpendicular a ele. Para este problema, isso não torna as coisas mais fáceis, uma vez que não há movimento no plano inclinado.

Do enunciado, retira-se a informação de que o block está em equilíbrio, pelo que não tem aceleração e as forças que em si atuam somam a zero. Este facto permite-nos escrever:

$$
N\sin(30^{\circ})-F_g=0 \qquad \implies \qquad N=\frac{F_g}{\sin(30^{\circ})}
$$

$$
N=\frac{(2.0\ kg)\cdot(9.8\ ms^{-2})}{\sin(30^{\circ})}=39.2\ N
$$

O resultado da força resultante horizontal também é nulo, logo podemos escrever:

$$
F-N\cos(30^{\circ})=0 \qquad \implies \qquad F=N\cos(30^{\circ})=33.9\ N
$$

\textbf{(b)} O resultado da força Normal foi encontrado anteriormente, $N=39.2\ N$.